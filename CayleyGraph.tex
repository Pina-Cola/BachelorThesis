% \documentclass[a4]{scrartcl}

\documentclass[12pt,a4paper, usenames, dvipsnames]{scrartcl}
\usepackage{graphicx}
\usepackage{xcolor}


\usepackage[ngerman]{babel}
\usepackage[utf8]{inputenc}
\usepackage{mathtools}
\usepackage{amsmath}
\usepackage{amssymb}
\usepackage{geometry}
\usepackage{scrpage2}
\usepackage{tikz}
\usepackage{float}
\usepackage{verbatim}

\usepackage{subfigure}
\usepackage[center]{caption}


\pagestyle{plain}
% \clearscrheadfoot
\pagenumbering {arabic} 

\usepackage[backend=bibtex,style=numeric, citestyle = numeric]{biblatex}
\addbibresource{quellen.bib}

\usepackage[babel,german=guillemets]{csquotes}




\geometry{
  paper=a4paper, % Change to letterpaper for US letter
  top=3cm, % Top margin
  left=2cm, % Left margin
  right=3cm, % Right margin
  %showframe, % Uncomment to show how the type block is set on the page
}

\pagenumbering {arabic} 









\begin{document}
{\huge Cayleygraph zum 2x2x2 Zauberwürfel} \\ \\ \\ \\
\  \\
Der Cayley Graph ist ein Graph, der Gruppen darstellt. \\
Laut dem Paper \glqq Applications of Cayley Graphs\grqq \  hat der Cayleygraph des 2x2x2 Würfels $3\ 674\ 160$ Knoten. \\
\\
\\
\fbox{\parbox{\linewidth}{
Meine Definition einer Gruppenoperation mit der Gruppe $(G_{2x2x2}, \circ)$ und der Menge $C$ (aus dem Proposal): \\ \\
Wenn der Zauberwürfel in einer Konfiguration $C=(\sigma, x)$ ist, wird der Würfel durch das Ausführen eines Zuges $M \in G_{2x2x2}$ in eine neue Konfiguration gebracht. Diese Konfiguration schreibe ich als $C \cdot M$. \\
\\
Definition einer Gruppenoperation mit der Gruppe $(G_{2x2x2}, \circ)$ und der Menge $C$:
\begin{itemize}
\item $\cdot: C \times G_{2x2x2} \rightarrow C$ mit $(c, g) \rightarrow c \cdot g $
\item $c \cdot N = x$ für alle $c \in C$ und das neutrale Element $N \in G_{2x2x2}$  
\item $c \cdot (u \circ v) = (c \cdot h) \cdot v$ für alle $u, v \in G_{2x2x2}$ und $c \in C$
\end{itemize}
\ \\
Angenommen der Würfel befindet sich in der Konfiguration $C$. Wenn nun der Zug $M_1 \in G_{2x2x2}$ ausgeführt wird, ist die neue Konfiguration des Würfels $C \cdot M_1$. Wenn nun noch ein weiterer Zug $M_2 \in G_{2x2x2}$ ausgeführt wird,ist die neue Konfiguration des Würfels $(C \cdot M_1) \cdot M_2$. \\
Anders gesagt: Der Würfel hat in Konfiguration $C$ gestartet und der Zug $M_1 M_2$ wurde ausgeführt. Man kann die neue Konfiguration alsl auch als $C \cdot (M_1 M_2)$ schreiben und somit gilt $(C \cdot M_1) \cdot M_2 = C \cdot (M_1 M_2)$. \\
\\
Wenn wir den leeren Zug $N$ ausführen, wird die Konfiguration des Würfels nicht verändert. Es gilt also $C \cdot N = C$. \\
\\
Bei Gruppenoperationen beeinflussen die Elemente einer Gruppe eine Menge. In diesem Fall beeinflussen die Züge des Würfels die Konfiguration des Würfels. \\
Es handelt sich hier um eine Rechtsoperation, da die Elemente der Gruppe rechts stehen.
}}




\end{document}




